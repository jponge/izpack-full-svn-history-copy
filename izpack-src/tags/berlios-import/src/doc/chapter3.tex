% Chapter 3
\chapter{Advanced Features}

% Ant Integration
\section{Ant Integration}

\IzPack can be easily integrated inside an Ant build process. To do so you
first need to tell Ant that you would like to use \IzPack :
\footnotesize
\begin{verbatim}
<!-- Allows us to use the IzPack Ant task -->
<taskdef name="izpack" classpath="${basedir}/lib/compiler.jar"
         classname="com.izforge.izpack.ant.IzPackTask"/>
\end{verbatim}
\normalsize

Don't forget to add \texttt{compiler.jar} to the classpath of the Ant process.\\

Then you can invoke \IzPack with the \texttt{izpack} task which takes the
following parameters :
\begin{itemize}

  \item \texttt{input} : the XML installation file
  \item \texttt{output} : the output jar installer file
  \item \texttt{installerType} : the installer type
  \item \texttt{baseDir} : the base directory to resolve the relative paths
  \item \texttt{izPackDir} : the \IzPack home directory.
  
\end{itemize}\

Here is a sample of the task invocation :\\
\footnotesize
\begin{verbatim}
<!-- We call IzPack -->
<echo message="Makes the installer using IzPack"/>
<izpack input="${dist.dir}/IzPack-install.xml"
        output="${dist.dir}/IzPack-install.jar"
        installerType="standard-kunststoff"
        basedir="${dist.dir}"
        izPackDir="${dist.dir}/"/>
\end{verbatim}
\normalsize

% Automated Installers
\section{Automated Installers}

When you conclude your installation with a FinishPanel, the user can
save the data for an automatic installation. With this data, he will be
able to run the same installation on another similar machine. In an
environment where many computers need to be supported this can save
\textsl{a lot} of time.\\

So run once the installation on a machine and save your automatic installation
data in \texttt{auto-install.xml} (that's just a sample). Then put this file in
the same directory as the installer on another machine. Run it with :\\
\texttt{java -jar installer.jar auto-install.xml}\\

It has reproduced the same installation :-)\\

% Picture on the Language Selection Dialog
\section{Picture on the Language Selection Dialog}

You can add a picture on the language selection dialog by adding the following
resource : \texttt{installer.langsel.img}. \textsl{GIF}, \textsl{JPEG} and
\textsl{PNG} pictures are supported starting from J2SE 1.3.\\

% Installer picture
\section{Picture in the installer}

It is possible to specify an optional picture to display on the left side of the
installer. To do this, you just have to define a resource whose id is
\texttt{Installer.image}. For instance,
\begin{verbatim}
<res id="Installer.image" src="nice-image.png" />
\end{verbatim}
will do that. If the resource isn't specified, no picture will be displayed at
all. \textsl{GIF}, \textsl{JPEG} and
\textsl{PNG} pictures are supported starting from J2SE 1.3.\\

% Native LAF installers
\section{Native-looking installers}

When using standard installers, it is possible to make them use the native
look and feel as provided by the JRE
(\texttt{UIManager.getNativeLookAndFeelClassName()}). To do that, just create
a 0-bytes file and add it as a resource with \texttt{useNativeLAF} as its ID.
That's all you have to do. If the JRE can't provide a native look and feel, then
the standard Metal look and feel will be used.\\

This feature should make happy a lot of users with the JDK 1.4.2 which
introduces native look and feel bindings for Windows XP and GTK+. I know it will
work like that for Windows XP but I'm not sure as far as GTK+ is concerned.\\

% Web Installers
\section{Web Installers}

The web installers allow your users to download a small installer that does not
contain the files to install. These files will be downloaded from a HTTP server
such as \textit{Apache HTTPD}. If you have many optional packs, this can save
people's resources. It's really easy : people download a small Jar
file containing the installer, they launch it and choose their
packages. Then the installer will get the files from another Jar file
located on a server. It's that simple.\\

Now suppose that you want to make an installer for your application
that you want to be named \texttt{install.jar}.
\begin{enumerate}
  \item open your favorite text editor and make a plain text file
  containing on the first line the URL to where you want to put the
  Jar file containing your packs, let's say for instance
  \url{http://www.mywebsite/myapp/install_web.jar}
  \item add this text file as a resource named
  \texttt{WebInstallers.url}
  \item compile your installer : you get \texttt{install.jar} and
  \texttt{install\_web.jar}
  \item copy \texttt{install\_web.jar} to
  \url{http://www.mywebsite/myapp/install_web.jar} and give your
  users \texttt{install.jar} for download.
\end{enumerate}\

That's all you need to make web installers.
Please note that the installation can look like frozen while the installer
grabs the server part.\\

% More Internationalization
\section{More Internationalization}

IzPack is available in several languages. However you might want to
internationalize some additional parts of your installer. In particular you
might want this for the *InfoPanel and *LicencePanel. This is actually pretty
easy to do. You just have to add one resource per localization, suffixed with the
ISO3 language code. At runtime these panels will try to load a localized version.\\

For instance let's suppose that we use a HTMLInfoPanel. Suppose that we have it
in English, French and German. We want to have a French text for french users.
Here we add a resource pointing to the French text whose name is
\texttt{HTMLInfoPanel.info\_fra}. And that's it ! English and German users (or
anywhere other than in France) will get the default text (denoted by 
\texttt{HTMLInfoPanel.info}) and the French users will get the French version.
Same thing for the other Licence and Info panels.\\

\noindent
\textit{To sum up :} add \texttt{\_<iso3 code>} to the resource name for
\texttt{InfoPanel}, \texttt{HTMLInfoPanel}, \texttt{LicencePanel} and
\texttt{HTMLLicencePanel}.\\
