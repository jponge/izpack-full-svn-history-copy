% Chapter 2
\chapter{Writing Installation XML Files}

% What you need
\section{What You Need}

\subsection{Your editor}

In order to write your XML installation files, you just need a plain
text editor. Of course it's always easier to work with color coded text,
so you might rather want to work with a text editor having such a
feature. Here is a list of free editors that work well :
\begin{itemize}
  \item Jext : \url{http://www.jext.org/}
  \item JEdit : \url{http://www.jedit.org/}
  \item classics like Vim and (X)Emacs.
\end{itemize}\

\subsection{Writing XML}

Though you might not know much about XML, you have certainly heard about
it. If you know XML you can skip this subsection as we will briefly
present how to use XML.\\

XML is a markup language, really close to HTML. If you've ever worked
with HTML the transition will be fast. However there are a few little
things to know. The markups used in XML have the following form :
\texttt{<markup>}. Each markup has to be closed somewhere with its
ending tag : \texttt{</markup>}. Each tag can contain text and other
markups. If a markup does not contain anything, it is just reported once
: \texttt{<markup/>}. A markup can contain attributes like :
\texttt{<markup attr1="123" attr2="hello !"/>}. Here is a sample of a
valid XML structure :
\footnotesize
\begin{verbatim}
<chapter title="Chapter 1">
  <section name="Introduction">
    <paragraph>
    This is the text of the paragraph number 1. It is available for the very low
    price of <price currency="dollar">1 000 000</price>.
    </paragraph>
  </section>
  <section name="xxx">
  xxx
  </section>
</chapter>
\end{verbatim}
\normalsize

You should be aware of the following common mistakes :
\begin{itemize}

  \item markups \textbf{are} case sensitive : \texttt{<markup>} is different
  from \texttt{<Markup>}.
  
  \item you \textbf{must} close the markups in the same order as you create them
  : \texttt{<m1><m2>(...)</m2></m1>} is right but 
  \texttt{<m1><m2>(...)</m1></m2>} is not.

\end{itemize}

Also, an XML file must start with the following header :\\
\texttt{<?xml version="1.0" encoding="iso-8859-1 standalone="yes" ?>}. The only
thing you should modify is the encoding (put here the one your text editor saves
your files to). The \texttt{standalone} attribute is not very important for
us.\\

This (brief !) introduction to XML was just meant to enable you to write
your installation specification. For a better introduction there are
plenty of books and articles/tutorials dealing with XML on the Internet,
in book stores, in magazines and so on.\\

% Built-in variables
\section{Variable Substitution}

During the installation process IzPack can substitute variables in
various places with real values. Obvious targets for variable
substitution are resource files and launch scripts, however you will
notice many more places where it is more powerful to use variables
rather then hard coded values. Wherever variables can be used it will
be explained in the documentation.\\

There are two types of variables:
\begin{itemize}
  \item Built-In variables. These are implemented in IzPack and are 
        all dynamic in nature. This means that the value of each
        variable depends on local conditions on the target system.
  \item Variables that you can define. You also define the value,
        which is fixed for a given installation file.
\end{itemize}

You define your own variables in the installation XML file with the
\texttt{<variable>} tag. How to do this is explained in detail later in
this chapter.\\

\textbf{Please note} that when using variables they must always appear
with a '\texttt{\$}' sign as the first character, even though they are
not defined this way.\\

\subsection{The Built-In Variables}
The following variables are built-in :
\begin{itemize}
  \item \texttt{\$INSTALL\_PATH} : the installation path on the
        target system, as chosen by the user.
  \item \texttt{\$JAVA\_HOME} : the \Java virtual machine home path
  \item \texttt{\$USER\_HOME} : the user's home directory path
  \item \texttt{\$USER\_NAME} : the user name
\end{itemize}\

\subsection{Parse Types}
Parse types apply only when replacing variables in text files. At places
where it might be necessary to specify a parse type, the documentation
will mention this. Depending on the parse type, IzPack will handle
special cases -such as escaping control characters- correctly. The
following parse types are available:
\begin{itemize}
  \item \texttt{plain} - use this type for plain text files, where no
        special substitution rules apply. All variables will be
        replaced with their respective values as is.
  \item \texttt{javaprop} - use this type if the substitution happens
        in a Java properties file. Individual variables might be
        modified to function properly within the context of Java
        property files.
  \item \texttt{xml} - use this type if the substitution happens in
        a XML file. Individual variables might be modified to function
        properly within the context of XML files.
\end{itemize}

% The IzPack elements
\section{The \IzPack Elements}

\noindent
\textit{When writing your installer XML files, it's a good idea to have a look
at the \IzPack installation DTD}.\\

\subsection{The Root Element \texttt{<installation>}}

The root element of an installation is \texttt{<installation>}. It takes
one required attribute : \texttt{version}. The attribute defines the
version of the XML file layout and is used by the compiler to identify
if it is compatible with the XML file. This should be set to $1.0$ for
the moment.\\

\subsection{The Information Element \texttt{<info>}}

This element is used to specify some general information for the installer. It
contains the following elements :
\begin{itemize}

  \item \texttt{<appname>} : the application name
  \item \texttt{<appversion>} : the application version
  \item \texttt{<url>} : the application official website url
  \item \texttt{<authors>} : specifies the author(s) of the application. It must contain
  at least one \texttt{<author>} element whose attributes are :
  \begin{itemize}
    \item \texttt{name} : the author's name
    \item \texttt{email} : the author's email
  \end{itemize}

\end{itemize}\

Here is an example of a typical \texttt{<info>} section :\\
\footnotesize
\begin{verbatim}
<info>
  <appname>Super extractor</appname>
  <appversion>2.1 beta 666</appversion>
  <url>http://www.superextractor.com/</url>
  <authors>
    <author name="John John Doo" email="jjd@jjd-mail.com"/>
    <author name="El Goyo" email="goyoman@mymail.org"/>
  </authors>
</info>
\end{verbatim}
\normalsize

\subsection{The Variables Element \texttt{<variables>}}

This element allows you to define variables for the variables
substitution system. Some variables are built-in, such as
\texttt{\$INSTALL\_PATH} (which is the installation path chosen by the
user). When you define a set of variables, you just have to place as
many \texttt{<variable>} tags in the file as needed. If you define a
variable named \texttt{VERSION} you need to type \$VERSION in the files
to parse. The variable substitutor will then replace it with the correct
value. One \texttt{<variable>} tag take the following attributes :
\begin{itemize}

  \item \texttt{name} : the variable name
  \item \texttt{value} : the variable value

\end{itemize}\

Here's a sample \texttt{<variables>} section :\\
\footnotesize
\begin{verbatim}
<variables>
  <variable name="app-version" value="1.4"/>
  <variable name="released-on" value="08/03/2002"/>
</variables>
\end{verbatim}
\normalsize

\subsection{The GUI Preferences Element \texttt{<guiprefs>}}

This element allows you to set the behavior of your installer GUI. This
information will not have any effect on the command-line installers that will be
available in future versions of \IzPack. The arguments to specify are :
\begin{itemize}

  \item \texttt{resizable} : takes \texttt{yes} or \texttt{no} and indicates
  wether the window size can be changed or not.
  \item \texttt{width} : sets the initial window width
  \item \texttt{height} : sets the initial window height
  
\end{itemize}\

Here's a sample :
\footnotesize
\begin{verbatim}
<guiprefs resizable="no" width="800" height="600"/>
\end{verbatim}
\normalsize

\subsection{The Localization Element \texttt{<locale>}}

This element is used to specify the language packs (langpacks) that you want to
use for your installer. You must set one \texttt{<langpack>} markup per
language. This markup takes the \texttt{iso3} parameter which specifies the iso3
language code.\\

Here's a sample :\\
\footnotesize
\begin{verbatim}
<locale>
  <langpack iso3="eng"/>
  <langpack iso3="fra"/>
  <langpack iso3="spa"/>
</locale>
\end{verbatim}\
\normalsize

The supported ISO3 codes are :
\begin{center}
\begin{tabular}{|l|l|}
\hline
\textit{ISO3 code} & \textit{Language} \\ \hline
cat & Catalunyan \\ \hline
deu & German \\ \hline
eng & English \\ \hline
fin & Finnish \\ \hline
fra & French \\ \hline
hun & Hungarian \\ \hline
jpn & Japanese \\ \hline
ned & Nederlands \\ \hline
pol & Polnish \\ \hline
por & Portuguese (Brazilian) \\ \hline
rus & Russian \\ \hline
spa & Spanish \\ \hline
swe & Swedish \\ \hline
ukr & Ukrainian \\ \hline
\end{tabular}\
\end{center}

\subsection{The Resources Element \texttt{<resources>}}

Several panels, such as the license panel and the shortcut panel,
require additional data to perform their task. This data is supplied in
the form of resources. This section describes how to specify them. Take
a look at each panel description to see if it might need any resources.
You have to set one \texttt{<res>} markup for each resource. Here are
the attributes to specify :
\begin{itemize}

  \item \texttt{src} : the path to the resource file which can be named freely
  of course (for instance \texttt{my-picture.jpg}).
  \item \texttt{id} : the resource id, depending on the needs of a particular panel
  \item \texttt{parse} : takes \texttt{yes} or \texttt{no} (default is
  \texttt{no}) - used to specify wether the resource must be parsed at the
  installer compilation time. For instance you could set the application version
  in a readme file used by \texttt{InfoPanel}.
  \item \texttt{type} : specifies the parse type. This makes sense only for a text
  resource  - the default is \texttt{plain}, other values are \texttt{javaprop,
  xml} (Java properties file and XML files)
  \item \texttt{encoding} : specifies the resource encoding if the receiver needs
  to know. This makes sense only for a text resource.

\end{itemize}\

Here's a sample :
\footnotesize
\begin{verbatim}
<resources>
  <res id="InfoPanel.info" src="doc/readme.txt" parse="yes"/>
  <res id="LicencePanel.licence" src="legal/License.txt"/>
</resources>
\end{verbatim}
\normalsize

\subsection{The Panels Element \texttt{<panels>}}

Here you tell the compiler which panels you want to use. They will
appear in the installer in the order in which they are listed in your
XML installation file. Take a look at the different panels in order to
find the ones you need. The \texttt{<panel>} markup takes a single
attribute \texttt{classname} which is the classname of the panel.\\

Here's a sample :
\footnotesize
\begin{verbatim}
<panels>
  <panel classname="HelloPanel"/>
  <panel classname="LicencePanel"/>
  <panel classname="TargetPanel"/>
  <panel classname="InstallPanel"/>
  <panel classname="FinishPanel"/>
</panels>
\end{verbatim}
\normalsize

\subsection{The Packs Section \texttt{<packs>}}

This is a crucial section as it is used to specify the files that need
to be installed. It contains the following XML elements :\\
\begin{itemize}

  \item \texttt{<pack>} : specifies a pack, takes the following attributes :
  \begin{itemize}
  
    \item \texttt{name} : the pack name
    \item \texttt{required} : takes \texttt{yes} or \texttt{no} and specifies
    wether the pack is optional or not.
    \item \texttt{os} : optional attribute that lets you make the pack targeted
    to a specific \textsl{operating system}, for instance \texttt{unix},
    \texttt{mac} and so on.
    
  \end{itemize}\
  The following tags are available for a \texttt{<pack>} markup :
  \begin{itemize}
  
    \item \texttt{<description>} : text describing the pack
    \item \texttt{<file>} : specifies a file to include, takes the following
    attributes :
    \begin{itemize}
    
      \item \texttt{src} : the file location (relative path) - if this is a
      directory its content will be added recursively
      \item \texttt{targetdir} : the destination directory, could be something
      like \texttt{\$INSTALL\_PATH/subdirX}
      \item \texttt{os} : can optionally specify a target operating system
      (\texttt{unix, windows, mac}) - this means that the file will only be
      installed on its target operating system
      \item \texttt{override} : if \texttt{true} then if the file is already
      installed, it will be overwritten. By default it is set to
      \texttt{false}.
    
    \end{itemize}\
    \item \texttt{<fileset>} : supports the Jakarta Ant powerful set syntax,
    takes the following parameters :
    \begin{itemize}
    
      \item \texttt{dir} : the base directory for the fileset (relative path)
      \item \texttt{targetdir} : the destination path, works like for
      \texttt{<file>}
      \item \texttt{casesensitive} : optionally lets you specify if the names
      are case-sensitive or not - takes \texttt{yes} or \texttt{no}
      \item \texttt{os} : specifies the operating system, works like for
      \texttt{<file>}
    
    \end{itemize}\
    You specify the files with  \texttt{<include>} and \texttt{<exclude>} tags
    that take the \texttt{name} parameter to specify the Ant-like pattern :
    \begin{itemize}    
      \item \texttt{**} : means any subdirectoy
      \item \texttt{*} : used as a wildcard.
    \end{itemize}
    Here are some examples of Ant patterns :
    \begin{itemize}
    
      \item \texttt{<include name="lib"/>} : will include \texttt{lib} and the
      subdirectories of \texttt{lib}
      \item \texttt{<exclude name="**/*.java"/>} : will exclude any file in any
      directory starting from the base path ending by \texttt{.java}
      \item \texttt{<include name="lib/*.jar"/>} : will include all the files
      ending by \texttt{.jar} in \texttt{lib}
      \item \texttt{<exclude name="lib/**/*FOO*"/>} : will exclude any file in
      any subdirectory starting from \texttt{lib} whose name contains
      \texttt{FOO}.
    
    \end{itemize}\
    \item \texttt{<parsable>} : used to specify the files for parsing by the
    variables substitutor, here are the attributes :
    \begin{itemize}
    
      \item \texttt{targetfile} : the file to parse, could be something like\\
      \texttt{\$INSTALL\_PATH/bin/launch-script.sh}
      \item \texttt{type} : specifies the type (same as for the resources) -
      the default is \texttt{plain}
      \item \texttt{encoding} : specifies the file encoding
    
    \end{itemize}\
    \item \texttt{<executable>} : a very useful thing if you need to execute
    something during the installation process. It can also be used to set
    the executable flag on Unix-like systems. Here are the attributes :
    \begin{itemize}
    
      \item \texttt{targetfile} : the file to run, could be something like\\
      \texttt{\$INSTALL\_PATH/bin/launch-script.sh}
      \item \texttt{class} : the class to run for a \Java program
      \item \texttt{type} : \texttt{bin} or \texttt{jar} (the default is
      \texttt{bin})
      \item \texttt{stage} : specifies when to launch : \texttt{postinstall}
      is just after the installation is done and the default value,
      \texttt{never} will never launch it (useful to set the +x flag on Unix).
      \texttt{uninstall} will launch the executable when the application
      is uninstalled. The executable is executed before any files are deleted.
      \item \texttt{failure} : specifies what to do when an error occurs :
      \texttt{abort} will abort the installation process, \texttt{ask} (default)
      will ask the user what to do and \texttt{warn} will just tell the user
      that something's wrong
    
    \end{itemize}\
    A \texttt{<args>} tag can also be specified in order to pass
    arguments to the executable:
    \begin{itemize}
    
      \item \texttt{<arg>} : passes the argument specified in the
      \texttt{value} attribute
    
    \end{itemize}\
    Finally it is possible to specify the target operating system with the
    \texttt{<os>} tag :
    \begin{itemize}
    
      \item \texttt{family} : \texttt{unix, windows, mac} to specify the
      operating system family
      \item \texttt{name} : the operating system name
      \item \texttt{version} : the operating system version
      \item \texttt{arch} : the operating system architecture (for instance the
      Linux kernel can run on i386, sparc, and so on)
    
    \end{itemize}\
  
  \end{itemize}\

\end{itemize}\

Here's an example installation file :
\footnotesize
\begin{verbatim}
<packs>
    <!-- The core files -->
    <pack name="Core" required="yes">
        <description>The IzPack core files.</description>
        <file targetdir="$INSTALL_PATH" src="bin"/>
        <file targetdir="$INSTALL_PATH" src="lib"/>
        <file targetdir="$INSTALL_PATH" src="legal"/>
        <file targetdir="$INSTALL_PATH" src="Readme.txt"/>
        <file targetdir="$INSTALL_PATH" src="Versions.txt"/>
        <file targetdir="$INSTALL_PATH" src="Thanks.txt"/>
        <parsable targetfile="$INSTALL_PATH/bin/izpack-fe"/>
        <parsable targetfile="$INSTALL_PATH/bin/izpack-fe.bat"/>
        <parsable targetfile="$INSTALL_PATH/bin/compile"/>
        <parsable targetfile="$INSTALL_PATH/bin/compile.bat"/>
        <executable targetfile="$INSTALL_PATH/bin/compile" stage="never"/>
        <executable targetfile="$INSTALL_PATH/bin/izpack-fe" stage="never"/>
    </pack>
    
    <!-- The documentation (1 directory) -->
    <pack name="Documentation" required="no">
        <description>The IzPack documentation (HTML and PDF).</description>
        <file targetdir="$INSTALL_PATH" src="doc"/>
    </pack>
</packs>
\end{verbatim}
\normalsize

\subsection{The Native Element \texttt{<native>}}

Use this if you want to use a feature that requires a native library.
The native libraries are placed under \texttt{bin/native/..}. There are 2
kinds of native libraries : the \IzPack libraries and the third-party
ones. The IzPack libraries are located at \texttt{bin/native/izpack},
you can place your own libraries at \texttt{bin/native/3rdparty}. The
markup takes the following attributes :
\begin{itemize}

  \item \texttt{type} : \texttt{izpack} or \texttt{3rdparty}
  \item \texttt{name} : the library filename

\end{itemize}\

Here's a sample :
\footnotesize
\begin{verbatim}
<native type="izpack" name="ShellLink.dll"/>
\end{verbatim}
\normalsize

\subsection{The jar Merging Element \texttt{<jar>}}

If you adapt \IzPack for your own needs, you might need to merge the
content of another jar file into the jar installer. For instance, this
could be a library that you need to merge. The \texttt{<jar>} markup
allows you to merge the raw content of another jar file, specified by
the \texttt{src} attribute.\\

A sample :
\footnotesize
\begin{verbatim}
<jar src="../nicelibrary.jar"/>
\end{verbatim}
\normalsize

% The panels
\section{The Available Panels}

In this section I will introduce the various panels available in IzPack.
The usage for most is pretty simple and described right here. The more
elaborate ones are explained in more detail in the \textit{Advanced
Features} chapter or in their own chapter. The panels are listed by
their class name. This is the name that must be used with the
\texttt{classname} attribute (case-sensitive).\\

\subsection{HelloPanel}

This panel welcomes the user by displaying the project name, the
version, the URL as well as the authors.\\

\subsection{InfoPanel and HTMLInfoPanel}

This is a kind of 'README' panel. It presents text of any length. The
text is specified by the \texttt{(HTML)InfoPanel.info} resource.\\

\subsection{LicencePanel and HTMLLicencePanel}

\noindent
\textit{\underline{Note :} there is a mistake in the name - it should be
LicensePanel. In France the word is Licence ... and one of my diploma is a
'Licence' so ...} :-)\\

These panels can prompt the user to acknowledge a license agreement. They block
unless the user selects the 'agree' option. To specify the license agreement
text you have to use the \texttt{(HTML)LicencePanel.licence} resource.\\

\subsection{PacksPanel}

Allows the user to select the packs he wants to install.\\

\subsection{ImgPacksPanel}

This is the same as above, but for each panel a different picture is
shown to the user. The pictures are specified with the resources
\texttt{ImgPacksPanel.img.x} where x stands for the pack number, the
numbers start from 0. Of course it's up to you to specify as many images
as needed and with correct numbers. For instance if you have 2 packs
\texttt{core} and \texttt{documentation} (in this order), then the resource for
\texttt{core} will be \texttt{ImgPacksPanel.img.0} and the resource for
\texttt{doc} will be \texttt{ImgPacksPanel.img.1}.\\

\subsection{TargetPanel}

This panel allows the user to select the installation path. It can be customized with
the following resources (they are text files containing the path) :
\begin{itemize}

  \item \texttt{TargetPanel.dir.f} where f stands for the family (\texttt{mac,
  macosx, windows, unix})
  \item \texttt{TargetPanel.dir} : the directory name, instead of the software
  to install name
  \item \texttt{TargetPanel.dir.d} where d is a "dynamic" name, as returned by
  the \Java virtual machine. You should write the name in lowercase and replace the
  spaces with underscores. For instance, you might want a different setting for
  Solaris and GNU/Linux which are both Unix-like systems. The resources would be
  \texttt{TargetPanel.dir.sunos, TargetPanel.dir.linux}. You should have a
  Unix-resource in case it wouldn't work though.

\end{itemize}\

\subsection{InstallPanel}

You should always have this one as it launches the installation process !\\

\subsection{XInfoPanel}

A panel showing text parsed by the variable substitutor. The text can be
specified through the \texttt{XInfoPanel.info} resource. This panel can
be useful when you have to show information after the installation
process is completed (for instance if the text contains the target
path).\\

\subsection{FinishPanel}

A ending panel, able to write automated installer information. For
details see the chapter on 'Advanced Features'.\\ 

\subsection{ShortcutPanel}

This panel is used to create desktop shortcuts. For details on using the
ShortcutPanel see the chapter 'Desktop Shortcuts'.
